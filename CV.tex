\documentclass{article}
\usepackage{titlesec}
\usepackage{geometry}[margin=2in]
\titleformat{\section}{\huge \bfseries}{}{}{}[\titlerule]
\titleformat{\subsection}{\large \bfseries}{$\bullet$}{}{ }
\titleformat{\subsubsection}{\bfseries}{}{}{}
\titlespacing{\section}{0em}{3em}{2em}
\titlespacing{\subsection}{0em}{2em}{1em}
\begin{document}
\title{Pratyush Kaware}
\author{\\Contact no. 9930800493 \\ Email : pratyushkaware99@gmail.com }
\maketitle
\section{Education}
    \subsection{Primary and Secondary Education :}
    Rustomjee International School , Borivali. (SSC Board) 
    \subsection{Higher Secondary :}
    PACE Junior College , Borivali. (HSC Board)
    \subsection{Graduation :}
    Currently studying at Sardar Patel Institute of Technology (B.Tech)
\section{\newpage Technical Skills}
    \subsection{Programming :}
    C , C++ , Java , Python , Assembly Language (Intel 8085 microprocessor) .
    \subsection{OS :}
    Used to UNIX environment.
    \subsection{Electronics :}
    Basic Electronics , worked with Raspberry Pi and Arduino .
\section{\newpage Projects}
    \subsection{Raspberry Pi 3 B+ :}
    \subsubsection{Remotely controlling Pi via SSH}
    Controlling Pi which is on the same network via SSH with my phone or any device connected on the same network with a terminal .
    ( For example : Turning LEDs on or off which were connected to the GPIO pins. ) 
    \subsubsection{Basic GPIO functions}
    Used General Purpose Input Output pins to provide a voltage and also used pulse width modulation to provide analog values .
    \subsubsection{Speech Recognition}
    Used Google Speech API with Python 3 to convert voice recorded by a microphone
    to text and executed commands related to GPIO pins based on the converted input ( like turning the LEDs on or off ).
    \subsection{Arduino :}
    \subsubsection{Used a 16x2 LCD display}
    Displaying text on the LCD .
    \subsubsection{Line Following Bot}
    Used an Arduino Nano with 3 IR sensors which gave input according to which the bot moved. 
    \subsection{Core Electronics :}
    \subsubsection{Variable DC Voltage Source}
    Built a variable DC voltage source using A transformer , Diodes , Voltage Regulator (LM317) , Capacitor and a Potentiometer . Can give DC voltage in the range of 1.25 V - 17 V .
    \subsubsection{Digital Counter}
    Built a 2-bit Digital Counter using clock (IC 555) , JK flipflops (IC 7473) , Binary to 7-segment Decoder (IC 4511 ) and a 7-segment display .
    
    
    
     
\end{document}